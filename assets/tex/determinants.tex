
\documentclass[10pt]{article}

\usepackage{enumerate}
\usepackage{amsmath,amsthm,amssymb,mathtools}
\usepackage{graphicx}
\usepackage{mycommands}
\usepackage{hyperref}

\newcommand{\eV}{\mtx{eV}}
\newcommand{\nm}{\mtx{nm}}
\newcommand{\varep}{\varepsilon}

\setlength{\oddsidemargin}{-0.25in}
\setlength{\evensidemargin}{-0.25in}
\setlength{\topmargin}{-0.25in}
\setlength{\headheight}{-0.25in}
\setlength{\headsep}{0in}
\setlength{\textwidth}{7.0in}
\setlength{\textheight}{9.5in}
\setlength{\parindent}{.25in}
\setlength{\parskip}{0in}

\begin{document}

\section{\label{sec:slnorm} Orthonormality of Slater determinant}
  \begin{align*}
    \ket \Psi
    &=
    \frac{1}{\sqrt{N!}} 
    \sum_{i_1\cdots i_N} \epsilon_{i_1 \cdots i_N}
    \phi_{i_1}(\mbf r_i,\sigma_i)
    \cdots
    \phi_{i_N}(\mbf r_N,\sigma_N) \\
    \implies
    \braket \Psi \Psi
    &=
    \frac1{N!} \int d\mbf r_1 \cdots d\mbf r_N
    \lt[\sum_{i_1\cdots i_N} \epsilon_{i_1 \cdots i_N} 
    \phi_{i_1}^*(\mbf r_1,\sigma_1) \cdots \rt]
    \lt[\sum_{j_1\cdots j_N} \epsilon_{j_1 \cdots j_N} 
    \phi_{j_1}(\mbf r_1,\sigma_1)   \cdots \rt] \\
    &=
    \frac 1{N!} \sum_{i_1\cdots i_N} \sum_{j_1\cdots j_N}
    \epsilon_{i_1 \cdots i_N}  \epsilon_{j_1 \cdots j_N}  
    \lt[\int d\mbf r_1 \phi_{i_1}^*(\mbf r_1,\sigma_1) \phi_{j_1}(\mbf r_1,\sigma_1)\rt]
    \cdots
    \lt[\int d\mbf r_N \phi_{i_N}^*(\mbf r_N,\sigma_N) \phi_{j_N}(\mbf r_N,\sigma_N)\rt]
  \end{align*} 
  Each integral is a $\delta_{ij}$ assuming the orbitals are orthonormal.
  \begin{align*} 
    \implies
    \braket \Psi \Psi
    &=
    \frac{1}{N!} \sum_{i_1 \cdots i_N} \epsilon_{i_1 \cdots i_N}^2
    =
    \frac{N!}{N!} = 1
  \end{align*} 

\section{\label{sec:overlap} Taking overlaps of Slater determinants in a basis}

  The expression from the \S~\ref{sec:slnorm} can be generalized to different determinants, and for an arbitrary basis:
  \begin{align}
    \braket{\Phi}{\Psi'}
    &=
    \frac 1{N!} \sum_{i_1\cdots i_N} \sum_{j_1\cdots j_N}
    \epsilon_{i_1 \cdots i_N}  \epsilon_{j_1 \cdots j_N}  
    \braket{\phi_{i_1}}{\psi_{j_1}}
    \cdots
    \braket{\phi_{i_N}}{\psi_{j_N}}
  \end{align}
  The ordering of the first index is irrelevant, and it cancels the prefactor:
  \begin{align}
    \braket{\Phi}{\Psi'}
    &=
    \sum_{j_1\cdots j_N}
    \epsilon_{j_1 \cdots j_N}  
    \braket{\phi_{1}}{\psi_{j_1}}
    \cdots
    \braket{\phi_{N}}{\psi_{j_N}}
  \end{align}
  This is the form of a determinant. In fact, for a determinant expressed in matrix form for a basis, $\ket{\alpha}$ with overlap $S_{\alpha,\alpha'}$, denoted $\Phi_{\alpha i} = \braket{\alpha}{\phi_i}$,
  \begin{align}
    \braket{\Phi}{\Psi'}
    &=
    \mrm{Det}\;(\Phi^T_{i,\alpha} S_{\alpha,\alpha'} \Psi_{\alpha',j})
  \end{align}
  This is because $\Phi^T_{i,\alpha} S_{\alpha,\alpha'} \Psi_{\alpha',j}$ results in a matrix of the overlaps:
  \begin{align}
    \Phi^T_{i,\alpha} S_{\alpha,\alpha'} \Psi_{\alpha',j}
    &=
    \begin{pmatrix}
      \braket{\phi_1}{\psi_1} & \cdots & \braket{\phi_1}{\psi_N} \\
      \vdots                 &        & \vdots \\
      \braket{\phi_N}{\psi_1} & \cdots & \braket{\phi_N}{\psi_N}
    \end{pmatrix}
  \end{align}

\section{\label{sec:slegy} Energy of a Slater determinant wavefunction}
  \begin{align*}
    \ket \Psi
    &=
    \frac{1}{\sqrt{N!}} 
    \sum_{\{i_\alpha\}}
    \epsilon_{i_1 \cdots i_N}
    \prod_{\alpha=1}^N \phi_{i_\alpha}(\mbf r_\alpha, \sigma_\alpha)
    \\
    \bra{\Phi} \hat H \ket{\Phi}
    &=
    \bra{\Phi} \hat T \ket{\Phi}
    +
    \bra{\Phi} \hat V_\mtx{ext} \ket{\Phi}
    +
    \bra{\Phi} \hat V_\mtx{int} \ket{\Phi}
    +
    E_{II}
    \\
    \bra{\Phi} \hat T \ket{\Phi}
    &=
    \frac{1}{N!} \sum_{\{i_\alpha\}} \sum_{\{j_\beta\}}
    \epsilon_{i_1 \cdots i_N} \epsilon_{j_1 \cdots j_N}
    \int d\mbf r_1 \cdots d\mbf r_N
    \prod_{\alpha=1}^N \phi_{i_\alpha}^*(\mbf r_\alpha, \sigma_\alpha)
    \lt[\sum_\gamma -\frac12 \del_\gamma^2 \rt]
    \prod_{\beta=1}^N \phi_{j_\beta}(\mbf r_\beta, \sigma_\beta)
    \\
  \end{align*}
  The steps that proceed are: commute all $\phi_{j_\beta}$ that commute with each
  $\del_\gamma^2$, leave the one that doesn't behind. The ones that commute will
  form an inner product that utilizes orthonormality to ensure $i_\alpha =
  j_{\alpha}$. Now we get the same redundancy as when computing the norm, and a
  simlar simplification follows.
  \begin{align*}
    \bra{\Phi} \hat T \ket{\Phi}
    &=
    -\frac12 
    \frac{1}{N!} \sum_{\{i_\alpha\}} \sum_{\{j_\beta\}}
    \epsilon_{i_1 \cdots i_N} \epsilon_{j_1 \cdots j_N}
    \sum_\gamma
    \int d\mbf r_1 \cdots d\mbf r_N
    \prod_{\alpha \ne \gamma} \phi_{i_\alpha}^*(\mbf r_\alpha, \sigma_\alpha)
    \prod_{\beta \ne \gamma} \phi_{j_\beta}(\mbf r_\beta, \sigma_\beta)
    \phi_{i_\gamma}^*(\mbf r_\gamma, \sigma_\gamma)
    \del_\gamma^2 \phi_{j_\gamma}(\mbf r_\gamma, \sigma_\gamma)
    \\
    &=
    -\frac12 
    \frac{1}{N!} \sum_{\{i_\alpha\}} \sum_{\{j_\beta\}}
    \epsilon_{i_1 \cdots i_N} \epsilon_{j_1 \cdots j_N}
    \sum_\gamma
    \prod_{\alpha \ne \gamma}
    \lt\{
    \int d\mbf r_\alpha 
      \phi_{i_\alpha}^*(\mbf r_\alpha, \sigma_\alpha)
      \phi_{j_\alpha}(\mbf r_\alpha, \sigma_\alpha)
    \rt\}
    \int d\mbf r_\gamma
    \phi_{i_\gamma}^*(\mbf r_\gamma, \sigma_\gamma)
    \del_\gamma^2 \phi_{j_\gamma}(\mbf r_\gamma, \sigma_\gamma)
    \\
    &=
    -\frac12 
    \frac{1}{N!} \sum_{\{i_\alpha\}} \sum_{\{j_\beta\}}
    \epsilon_{i_1 \cdots i_N} \epsilon_{j_1 \cdots j_N}
    \sum_\gamma
    \prod_{\alpha \ne \gamma}
    \delta_{i_\alpha,j_\alpha}
    \int d\mbf r_\gamma
    \phi_{i_\gamma}^*(\mbf r_\gamma, \sigma_\gamma)
    \del_\gamma^2 \phi_{j_\gamma}(\mbf r_\gamma, \sigma_\gamma)
  \end{align*}
  Thus the sum over $j_\beta$ is consumed, setting all of $i_\alpha = j_\alpha$
  for $\alpha \ne \gamma$, however this also determines that $i_\gamma = j_\gamma$
  by process of elimination. This produces a sum of $N!$ identical terms.
  \begin{align*}
    \implies
    \bra{\Phi} \hat T \ket{\Phi}
    &=
    -\frac12 
    \sum_\gamma
    \int d\mbf r_\gamma
    \phi_\gamma^*(\mbf r_\gamma, \sigma_\gamma)
    \del_\gamma^2 \phi_\gamma(\mbf r_\gamma, \sigma_\gamma)
  \end{align*}
  Thus the kinetic energy of the product wavefuntion is simply the sum of the
  kinetic energies of the orbitals. This will occur the same for any one-body
  property, since for these operators, we can form the inner product that sets the
  $i_\alpha = j_\alpha$ for all $\alpha$.
  \begin{align*}
    \implies
    \bra{\Phi}[ \hat T + \hat V_\mtx{ext} ]\ket{\Phi}
    &=
    \sum_i \int d\mbf r~
    \phi_i^*(\mbf r,\sigma_i)
    \lt[
      -\frac12 \del^2 + V_\mtx{ext}(\mbf r)
    \rt]
    \phi_i(\mbf r,\sigma_i)
  \end{align*}

  For two body operators, something similar happens. Consider $\hat V_\mtx{int}$.
  \begin{align*}
    \bra{\Phi} \hat V_\mtx{int} \ket{\Phi}
    &=
    \frac{1}{N!} \sum_{\{i_\alpha\}} \sum_{\{j_\beta\}}
    \epsilon_{i_1 \cdots i_N} \epsilon_{j_1 \cdots j_N}
    \int d\mbf r_1 \cdots d\mbf r_N
    \prod_{\alpha=1}^N \phi_{i_\alpha}^*(\mbf r_\alpha, \sigma_\alpha)
    \lt[\frac 12 \sum_{\gamma \ne \delta} \frac 1{|\mbf r_\gamma - \mbf r_\delta|} \rt]
    \prod_{\beta=1}^N \phi_{j_\beta}(\mbf r_\beta, \sigma_\beta)
    \\
    &=
    \frac12 
    \frac{1}{N!} \sum_{\{i_\alpha\}} \sum_{\{j_\beta\}}
    \epsilon_{i_1 \cdots i_N} \epsilon_{j_1 \cdots j_N}
    \sum_{\gamma \ne \delta}
    \prod_{\alpha \ne \gamma,\delta}
    \lt\{
    \int d\mbf r_\alpha 
      \phi_{i_\alpha}^*(\mbf r_\alpha, \sigma_\alpha)
      \phi_{j_\alpha}(\mbf r_\alpha, \sigma_\alpha)
    \rt\}
    \\
    &\times
    \int d\mbf r_\gamma d\mbf r_\delta
    \lt[
    \frac 
    {
      \phi_{i_\gamma}^*(\mbf r_\gamma, \sigma_\gamma)
      \phi_{i_\delta}^*(\mbf r_\delta, \sigma_\delta)
      \phi_{j_\gamma}(\mbf r_\gamma, \sigma_\gamma)
      \phi_{j_\delta}(\mbf r_\delta, \sigma_\delta)
    }
    {|\mbf r_\gamma - \mbf r_\delta|} \rt]
    \\
    &=
    \frac12 
    \frac{1}{N!} \sum_{\{i_\alpha\}} \sum_{\{j_\beta\}}
    \epsilon_{i_1 \cdots i_N} \epsilon_{j_1 \cdots j_N}
    \sum_{\gamma \ne \delta}
    \prod_{\alpha \ne \gamma,\delta}
    \delta_{i_\alpha,j_\alpha}
    \int d\mbf r_\gamma d\mbf r_\delta
    \lt[
    \frac 
    {
      \phi_{i_\gamma}^*(\mbf r_\gamma, \sigma_\gamma)
      \phi_{i_\delta}^*(\mbf r_\delta, \sigma_\delta)
      \phi_{j_\gamma}(\mbf r_\gamma, \sigma_\gamma)
      \phi_{j_\delta}(\mbf r_\delta, \sigma_\delta)
    }
    {|\mbf r_\gamma - \mbf r_\delta|} \rt]
  \end{align*}
  Now for $\alpha \ne \gamma, \delta$, $i_\alpha = j_\alpha$, but this time we
  cannot further deduce that $i_\gamma = j_\gamma$ since there are two options
  instead: $i_\gamma = j_\gamma$ and $i_\delta = j_\delta$, or $i_\gamma =
  j_\delta$ and $i_\delta = j_\gamma$. The sum over $j_\alpha$ is consumed by the
  delta functions (keeping in mind that observation). In the case that $i_\gamma =
  j_\gamma$, the Levi-Civita symbol becomes squared and the term is positive. In
  the other case, they differ by an exchange of indices, thus the product is negative.
  \begin{align*}
    \implies
    \bra{\Phi} \hat V_\mtx{int} \ket{\Phi}
    &=
    \frac12 
    \frac{1}{N!} \sum_{\{i_\alpha\}}
    \sum_{\gamma \ne \delta}
    \int d\mbf r_\gamma d\mbf r_\delta
    \lt\{
      \lt[
      \frac 
      {
        \phi_{i_\gamma}^*(\mbf r_\gamma, \sigma_\gamma)
        \phi_{i_\delta}^*(\mbf r_\delta, \sigma_\delta)
        \phi_{i_\gamma}(\mbf r_\gamma, \sigma_\gamma)
        \phi_{i_\delta}(\mbf r_\delta, \sigma_\delta)
      }
      {|\mbf r_\gamma - \mbf r_\delta|} \rt]
    \rt.
      \\
      &-
    \lt.
      \lt[
      \frac 
      {
        \phi_{i_\gamma}^*(\mbf r_\gamma, \sigma_\gamma)
        \phi_{i_\delta}^*(\mbf r_\delta, \sigma_\delta)
        \phi_{i_\delta}(\mbf r_\gamma, \sigma_\gamma)
        \phi_{i_\gamma}(\mbf r_\delta, \sigma_\delta)
      }
      {|\mbf r_\gamma - \mbf r_\delta|} \rt]
    \rt\}
    \\
    &=
    \frac12 
    \sum_{\gamma \ne \delta}
    \int d\mbf r_\gamma d\mbf r_\delta
    \lt[
    \frac 
    {
      \phi_{\gamma}^*(\mbf r_\gamma, \sigma_\gamma)
      \phi_{\delta}^*(\mbf r_\delta, \sigma_\delta)
      \phi_{\gamma}(\mbf r_\gamma, \sigma_\gamma)
      \phi_{\delta}(\mbf r_\delta, \sigma_\delta)
    }
    {|\mbf r_\gamma - \mbf r_\delta|} \rt]
      \\
      &-
    \frac12 
    \sum_{\gamma \ne \delta}
    \int d\mbf r_\gamma d\mbf r_\delta
    \lt[
    \frac 
    {
      \phi_{\gamma}^*(\mbf r_\gamma, \sigma_\gamma)
      \phi_{\delta}^*(\mbf r_\delta, \sigma_\delta)
      \phi_{\delta}(\mbf r_\gamma, \sigma_\gamma)
      \phi_{\gamma}(\mbf r_\delta, \sigma_\delta)
    }
    {|\mbf r_\gamma - \mbf r_\delta|} \rt]
    \\
    &=
    \frac12 
    \sum_{\gamma, \delta}
    \int d\mbf r_\gamma d\mbf r_\delta
    \lt[
    \frac 
    {
      \phi_{\gamma}^*(\mbf r_\gamma, \sigma_\gamma)
      \phi_{\delta}^*(\mbf r_\delta, \sigma_\delta)
      \phi_{\gamma}(\mbf r_\gamma, \sigma_\gamma)
      \phi_{\delta}(\mbf r_\delta, \sigma_\delta)
    }
    {|\mbf r_\gamma - \mbf r_\delta|} \rt]
      \\
      &-
    \frac12 
    \sum_{\gamma, \delta}
    \int d\mbf r_\gamma d\mbf r_\delta
    \lt[
    \frac 
    {
      \phi_{\gamma}^*(\mbf r_\gamma, \sigma_\gamma)
      \phi_{\delta}^*(\mbf r_\delta, \sigma_\delta)
      \phi_{\delta}(\mbf r_\gamma, \sigma_\gamma)
      \phi_{\gamma}(\mbf r_\delta, \sigma_\delta)
    }
    {|\mbf r_\gamma - \mbf r_\delta|} \rt]
  \end{align*}
  Putting this analysis together gives us our result.

\section{\label{sec:hfeq} Derivation of the Hartree-Fock equations}
  I suppress spin degree of freedom for simplicity.

  We appoach it via the method of Lagrange multipliers. We vary the $\phi_i^*(\mbf
  r)$ subject to the constraints $\braket{\phi_i}{\phi_i} = 1$. Basic theory of
  Lagrange multipliers requires that
  \begin{align*}
    \delta \bra{\Phi} H \ket{\Phi} - \sum_i \delta \braket{\phi_i}{\phi_i}
    &=
    0
  \end{align*}
  So varying $\phi^*_i$,
  \begin{align*}
    \delta \bra{\Phi} \hat H \ket{\Phi}
    &=
    \sum_i \int d\mbf r
    \delta \phi_i^*(\mbf r)
    \lt(\hat T + \hat V_\mtx{ext}(\mbf r)\rt)
    \phi_i(\mbf r)
    \\
    &+
    \frac 12 \sum_{ij} \int d\mbf r d\mbf r'
    \frac{\delta \phi_i^*(\mbf r) \phi_j^*(\mbf r') \phi_i(\mbf r) \phi_j(\mbf r')}
    {|\mbf r - \mbf r'|}
    +
    \frac 12 \sum_{ij} \int d\mbf r d\mbf r'
    \frac{\phi_i^*(\mbf r) \delta \phi_j^*(\mbf r') \phi_i(\mbf r) \phi_j(\mbf r')}
    {|\mbf r - \mbf r'|}
    \\
    &-
    \frac 12 \sum_{ij} \int d\mbf r d\mbf r'
    \frac{\delta \phi_i^*(\mbf r) \phi_j^*(\mbf r') \phi_j(\mbf r) \phi_i(\mbf r')}
    {|\mbf r - \mbf r'|}
    -
    \frac 12 \sum_{ij} \int d\mbf r d\mbf r'
    \frac{\phi_i^*(\mbf r) \delta \phi_j^*(\mbf r') \phi_j(\mbf r) \phi_i(\mbf r')}
    {|\mbf r - \mbf r'|}
  \end{align*}
  The right side two-body terms are identical to the left side up to renaming the
  summation index and integration variables.
  \begin{align*}
    \delta \bra{\Phi} \hat H \ket{\Phi}
    &=
    \sum_i \int d\mbf r
    \delta \phi_i^*(\mbf r)
    \lt(\hat T + \hat V_\mtx{ext}(\mbf r)\rt)
    \phi_i(\mbf r)
    \\
    &+
    \sum_{ij} \int d\mbf r d\mbf r'
    \frac{\delta \phi_i^*(\mbf r) \phi_j^*(\mbf r') \phi_i(\mbf r) \phi_j(\mbf r')}
    {|\mbf r - \mbf r'|}
    -
    \sum_{ij} \int d\mbf r d\mbf r'
    \frac{\delta \phi_i^*(\mbf r) \phi_j^*(\mbf r') \phi_j(\mbf r) \phi_i(\mbf r')}
    {|\mbf r - \mbf r'|}
    \\
    &=
    \sum_i \int d\mbf r \delta \phi_i^*(\mbf r) \lt\{
      \lt(\hat T + \hat V_\mtx{ext} \rt)
      + \sum_j \int d\mbf r' 
      \frac{\phi_j^*(\mbf r') \phi_i(\mbf r) \phi_j(\mbf r')}
      {|\mbf r - \mbf r'|}
      - \sum_j \int d\mbf r' 
      \frac{\phi_j^*(\mbf r') \phi_j(\mbf r) \phi_i(\mbf r')}
      {|\mbf r - \mbf r'|}
    \rt\}
  \end{align*}
  Thus,
  \begin{align*}
    &\delta \bra{\Phi} H \ket{\Phi} - \sum_i \delta \braket{\phi_i}{\phi_i}
    = 0
    \\
    \iff
    &\sum_i \int d\mbf r \delta \phi_i^*(\mbf r) \lt\{
      \lt(\hat T + \hat V_\mtx{ext} \rt) \phi_i(\mbf r)
      + \sum_j \int d\mbf r' 
      \frac{\phi_j^*(\mbf r') \phi_i(\mbf r) \phi_j(\mbf r')}
      {|\mbf r - \mbf r'|}
      - \sum_j \int d\mbf r' 
      \frac{\phi_j^*(\mbf r') \phi_j(\mbf r) \phi_i(\mbf r')}
      {|\mbf r - \mbf r'|}
      -
      \vep_i \phi_i(\mbf r)
    \rt\}
    = 0
  \end{align*}
  Since $\delta \phi_i^*(\mbf r)$ is arbitrary, the term in the brackets must be
  zero for all $i$. These are the Hartree-Fock equations.

\section{\label{sec:koop} Proof of Koopman's theorem}
  Here are the Hartree-Fock equations written more explicitly:
  \begin{align}
    \label{eq:hfeq}
    \lt[
      -\frac12 \del^2 + V_\mtx{ext}
      + \sum_{j,\sigma_j} \int d\mbf r' 
      \frac{\phi_j^{\sigma_j *}(\mbf r') \phi_j^{\sigma_j}(\mbf r')}{|\mbf r - \mbf r'|}
    \rt] \phi_i^{\sigma} (\mbf r)
    - \sum_j \int d\mbf r' 
    \frac{\phi_j^{\sigma_j *}(\mbf r') \phi_i^{\sigma}(\mbf r')}{|\mbf r - \mbf r'|} \phi_j^\sigma(\mbf r)
    =
    \varepsilon_i^\sigma \phi_i^\sigma(\mbf r)
  \end{align}
  Taking an inner product of (\ref{eq:hfeq}) with $\bra{\phi_i}$,
  \begin{align*}
    \varepsilon_i^\sigma
    &=
    \int d\mbf r~\phi_i^*(\mbf r) \lt\{
      \lt[
        -\frac12 \del^2 + V_\mtx{ext}
        + \sum_{j,\sigma_j} \int d\mbf r' 
        \frac{\phi_j^{\sigma_j *}(\mbf r') \phi_j^{\sigma_j}(\mbf r')}{|\mbf r - \mbf r'|}
      \rt] \phi_i^{\sigma} (\mbf r)
      -
      \sum_j \int d\mbf r' 
      \frac{\phi_j^{\sigma_j *}(\mbf r') \phi_i^{\sigma}(\mbf r')}{|\mbf r - \mbf r'|} \phi_j^\sigma(\mbf r)
    \rt\}
    \\
    &=
    \int d\mbf r~ \phi_i^*(\mbf r) \lt[
      \hat T + V_\mtx{ext}(\mbf r)
    \rt] \phi(\mbf r)
    +
    \sum_j \int d\mbf r~d\mbf r'
    \frac{\phi_i^*(\mbf r) \phi_j^*(\mbf r') \phi_i(\mbf r) \phi_j(\mbf r')}
    {|\mbf r - \mbf r'|}
    -
    \sum_j \int d\mbf r~d\mbf r'
    \frac{\phi_i^*(\mbf r) \phi_j^*(\mbf r') \phi_j(\mbf r) \phi_i(\mbf r')}
    {|\mbf r - \mbf r'|}
  \end{align*}
  On the other hand, consider the Slater determinant with the $k$th orbital
  missing, denoted $\ket{\Phi_{N-1}^{(k)}}$. (See eq. \ref{eq:hfengy}).
  \begin{align*}
    \bra{\Phi_{N-1}^{(k)}} \hat H \ket{\Phi_{N-1}^{(k)}}
    &=
    \sum_{i \ne k} \int d\mbf r~\phi_i(\mbf r) \lt[
      \hat T + V_\mtx{ext}(\mbf r) 
    \rt] \phi_i(\mbf r)
    \\
    &+
    \frac 12 \sum_{i,j\ne k}\int d\mbf r~d\mbf r'~
    \frac{\phi_i^*(\mbf r) \phi_j^*(\mbf r') \phi_i(\mbf r) \phi_j(\mbf r')}
    {|\mbf r - \mbf r'|}
    \\
    &-
    \frac 12 \sum_{i,j\ne k}\int d\mbf r~d\mbf r'~
    \frac{\phi_i^*(\mbf r) \phi_j^*(\mbf r') \phi_j(\mbf r) \phi_i(\mbf r')}
    {|\mbf r - \mbf r'|}
  \end{align*}
  The one-body terms come readily, since they are simply missing one term in the
  sum (when $i = k$). The two-body terms have two series of terms for both the
  direct and exchange terms that are missing, when $i=k$ and when $j=k$. I'll
  focus on the direct terms, since the exchange follow for identical reasons.
  \begin{align*}
    \implies
    \bra{\Phi_{N-1}^{(k)}} \hat H \ket{\Phi_{N-1}^{(k)}} - \bra{\Phi} \hat H \ket{\Phi}
    &=
    -\int d\mbf r~\phi_k(\mbf r) \lt[
      \hat T + V_\mtx{ext}(\mbf r) 
    \rt] \phi_k(\mbf r)
    \\
    &-
    \frac 12 \sum_j \int d\mbf r~d\mbf r'
    \frac{\phi_k^*(\mbf r) \phi_j^*(\mbf r') \phi_k(\mbf r) \phi_j(\mbf r')}
    {|\mbf r - \mbf r'|}
    \\
    &-
    \frac 12 \sum_i \int d\mbf r~d\mbf r'
    \frac{\phi_i^*(\mbf r) \phi_k^*(\mbf r') \phi_i(\mbf r) \phi_k(\mbf r')}
    {|\mbf r - \mbf r'|}
    \\
    &+
    \mtx{Exchange~terms}
  \end{align*}
  Now the two direct terms will be identical if we exchange both $\mbf r
  \leftrightarrow \mbf r'$ in the second integral. The same simplification follows
  for the exchange term which I neglected to write out.
  \begin{align*}
    \implies
    \bra{\Phi_{N-1}^{(k)}} \hat H \ket{\Phi_{N-1}^{(k)}} - \bra{\Phi} \hat H \ket{\Phi}
    &=
    -\int d\mbf r~\phi_k(\mbf r) \lt[
      \hat T + V_\mtx{ext}(\mbf r) 
    \rt] \phi_k(\mbf r)
    \\
    &-
    \sum_j \int d\mbf r~d\mbf r'
    \frac{\phi_k^*(\mbf r) \phi_j^*(\mbf r') \phi_k(\mbf r) \phi_j(\mbf r')}
    {|\mbf r - \mbf r'|}
    \\
    &+
    \sum_j \int d\mbf r~d\mbf r'
    \frac{\phi_k^*(\mbf r) \phi_j^*(\mbf r') \phi_j(\mbf r) \phi_k(\mbf r')}
    {|\mbf r - \mbf r'|}
  \end{align*}
  which now matches the right hand side of the expression for $\varepsilon_k$.
  This completes the proof for subtraction of orbitals from the Slater
  determinant, the same sequence of steps works with adding an orbital, since now
  $\bra \Phi \hat H \ket \Phi$ does not contain the added orbital, and hence the
  result switches sign.

\end{document}


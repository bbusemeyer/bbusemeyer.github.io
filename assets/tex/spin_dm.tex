
\documentclass[10pt]{article}

\usepackage{enumerate}
\usepackage{amsmath,amsthm,amssymb,mathtools}
\usepackage{graphicx}
\usepackage{mycommands}
\usepackage{hyperref}

\newcommand{\eV}{\mtx{eV}}
\newcommand{\nm}{\mtx{nm}}
\newcommand{\varep}{\varepsilon}

\setlength{\oddsidemargin}{-0.25in}
\setlength{\evensidemargin}{-0.25in}
\setlength{\topmargin}{-0.25in}
\setlength{\headheight}{-0.25in}
\setlength{\headsep}{0in}
\setlength{\textwidth}{7.0in}
\setlength{\textheight}{9.5in}
\setlength{\parindent}{.25in}
\setlength{\parskip}{0in}

\begin{document}

\title{Computing spin correlations with 2-body density matrix\\ and other misc notes}
\maketitle

Operators are written like variables for simplicity. What is or isn't an operator should be clear from context.

\section{Computing spin correlations with 2-body density matrix}

The prerequisites are that 
\begin{align}
  &
  S_\pm = S_x \pm i S_y,
  &&
  S_z = \frac 12 (c_\up^\dag c_\up - c_\dn^\dag c_\dn)
  &
\end{align}
which can be found by Wikipedia, looking for ``spin operator". 
Using this, 
\begin{align}
  &
  S_x = \frac 12 (S_+ + S_-)
  &&
  S_y = \frac 1{2i} (S_+ - S_-)
  &
\end{align}
You can also derive this by taking $S_\alpha = c^\dag_\sigma \tau_{\alpha,\sigma\sigma'} c_{\sigma'}$, where $\tau_{\alpha,\sigma\sigma'}$ are the Pauli matrices.

The rest is careful algebra after plugging these in. 
\begin{align}
  S_i \cdot S_j
  &=
  S_{ix} S_{jx} 
  +
  S_{iy} S_{jy} 
  +
  S_{iz} S_{jz} 
  \\
  &=
  \frac 12 (
    S_{i+} S_{j-} + S_{i-} S_{j+}
  )
  + 
  S_{iz} S_{jz}
  \\
  &=
  \frac 12 \lt(
    c_{i\up}^\dag
    c_{i\dn}
    c_{j\dn}^\dag
    c_{j\up}
    +
    c_{i\dn}^\dag
    c_{i\up}
    c_{j\up}^\dag
    c_{j\dn}
  \rt)
  +
  \frac 14 \lt(
    c_{i\up}^\dag
    c_{i\up}
    c_{j\up}^\dag
    c_{j\up}
    +
    c_{i\dn}^\dag
    c_{i\dn}
    c_{j\dn}^\dag
    c_{j\dn}
    -
    c_{i\dn}^\dag
    c_{i\dn}
    c_{j\up}^\dag
    c_{j\up}
    -
    c_{i\up}^\dag
    c_{i\up}
    c_{j\dn}^\dag
    c_{j\dn}
  \rt)
  \\
  &=
  -\frac 12 \lt(
    c_{i\up}^\dag
    c_{j\dn}^\dag
    c_{i\dn}
    c_{j\up}
    +
    c_{j\up}^\dag
    c_{i\dn}^\dag
    c_{j\dn}
    c_{i\up}
  \rt)
  -
  \frac 14 \lt(
    c_{i\up}^\dag
    c_{j\up}^\dag
    c_{i\up}
    c_{j\up}
    +
    c_{i\dn}^\dag
    c_{j\dn}^\dag
    c_{i\dn}
    c_{j\dn}
    +
    c_{j\up}^\dag
    c_{i\dn}^\dag
    c_{i\dn}
    c_{j\up}
    +
    c_{i\up}^\dag
    c_{j\dn}^\dag
    c_{j\dn}
    c_{i\up}
  \rt)
\end{align}
The last step is rearranging the operators according to the chemistry notation~(\ref{eq:chem}).
Using that notation,
\begin{align}
  S_i \cdot S_j
  &=
  -\frac 12 \lt(
    \rho_{ijji}^{\up\dn}
    +
    \rho_{jiij}^{\up\dn}
  \rt)
  -
  \frac 14 \lt(
    \rho_{ijji}^{\up\up}
    +
    \rho_{ijji}^{\dn\dn}
    +
    \rho_{jjii}^{\up\dn}
    +
    \rho_{iijj}^{\up\dn}
  \rt)
\end{align}

\section{Misc notes when working on this}

\verb|PySCF| uses the ``chemistry notation'' when storing the 2-body density matrix. In particular,
\begin{align}
  \label{eq:chem}
  \rho^{\up \dn}_{ijkl}
  =
  \abr{c_{i\up}^\dag c_{k\dn}^\dag c_{l\dn} c_{j\up}}
\end{align}
This is documented in, for example, the FCI \verb|make_rdm12s| docstring. 
This notation has the advantage that when $k = j$ the center becomes a density operator.

Using $\{c_i^\dag,c_j\} = 0$, it differs by a sign and \verb|swapaxes| from the physics notation (TODO verify this...it's been a while and this is from memory):
\begin{align}
  \rho^{\up \dn}_{ijkl}
  =
  \abr{c_{i\up}^\dag c_{j\dn}^\dag c_{k\up} c_{l\dn}}
\end{align}

\end{document}


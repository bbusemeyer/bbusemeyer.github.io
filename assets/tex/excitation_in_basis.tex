
\documentclass[10pt]{article}

\usepackage{enumerate}
\usepackage{amsmath,amsthm,amssymb,mathtools}
\usepackage{graphicx}
\usepackage{mycommands}
\usepackage{hyperref}

\newcommand{\eV}{\mtx{eV}}
\newcommand{\nm}{\mtx{nm}}
\newcommand{\varep}{\varepsilon}

\setlength{\oddsidemargin}{-0.25in}
\setlength{\evensidemargin}{-0.25in}
\setlength{\topmargin}{-0.25in}
\setlength{\headheight}{-0.25in}
\setlength{\headsep}{0in}
\setlength{\textwidth}{7.0in}
\setlength{\textheight}{9.5in}
\setlength{\parindent}{.25in}
\setlength{\parskip}{0in}

\begin{document}

\section{Doing MO excitation for a hamiltonian in a localized basis. }

I first came across this when trying to form singles excitations for a embedded hamiltonian for AFQMC. 
I figured I'd write down the notes so I wouldn't need to rederive this again. 

\subsection{Definitions}

The elements of a localized basis will be $\ket{\alpha} = a^\dag_\alpha \ket{0}$.
The MO basis for a SCF calculation is $\ket{\phi_i} = c_i^\dag \ket{0}$. 

\subsection{Occupying the LOs for a representation of the ground state SCF determinant}

The localized orbitals are defined in terms of occupied and unoccupied MOs, so the first $N_\mtx{elec}$ LO orbitals, $\ket{\alpha}$, are rotations of occupied MOs.
Therefore the SCF ground state is equivilent to a determinant that is constructed from occupying the occupied local orbitals.
This state would look something like:
\begin{align}
  \ket{\Phi^{(0)}} 
  &\sim
  \begin{pmatrix}
    1 &   &        &  &  &   \\
      & 1 &        &  &  &   \\
      &   & \ddots &  &  &   \\
      &   &  & \ddots  &  &   \\
      &   &        &  &  & 1  \\
    0 & \cdots &  &  &  & 0 \\
    \vdots & \ddots & \ddots & \ddots & & \vdots \\
    0 & \cdots & & & & 0
  \end{pmatrix}
  =
  \Phi_{\alpha \alpha'}
  = 
  \braket{\alpha}{\alpha'}
\end{align}
The shape of the matrix is $N_\mtx{basis} \times N_\mtx{occ}$.
There's one like that for each spin.
The python command is \\\verb|np.concatenate( (np.eye(maxN),np.zeros((nBasis-maxN,maxN))) )|.
In this case the ``MOs" \textbf{are} the LOs.

This construction of the ground state is more convenient than rotating the SCF ground state into a localized basis, because when freezing the localized basis by location, many occupied MOs will overlap with a single LO.
You know which LO are active, but each MO has some overlap with an occupied LO. 
When you freeze the system, you can truncate the hamiltonian, but it is difficult to know which rows or columns to remove from a SCF determinent in the MO basis. 

\subsection{Performing an excitation on a different basis}

Essentially a HOMO-LUMO excitation would be something like:
\begin{align}
  \ket{\Phi^{(1)}} 
  &=
  c_{i+1}^\dag c_i \ket{\Phi^{(0)}}
\end{align}
Changing basis from the MOs to the LOs can be found:
\begin{align}
  c_i^\dag \ket{0}
  &=
  \ket{\phi_i}
  =
  \sum_\alpha 
  \ket{\alpha} \braket{\alpha}{\phi_i}
  =
  \sum_\alpha 
  a_\alpha^\dag \ket{0} \braket{\alpha}{\phi_i}
  \\
  &=
  \lt(
    \sum_\alpha 
    a_\alpha^\dag \braket{\alpha}{\phi_i}
  \rt)
  \ket{0}
\end{align}
So:
\begin{align}
  & c_i^\dag = \sum_\alpha a_\alpha^\dag \braket{\alpha}{\phi_i} &
  & c_i = \sum_\alpha \braket{\phi_i}{\alpha} a_\alpha &
  \\
\end{align}
Then:
\begin{align}
  c_{i+1}^\dag c_i 
  \ket{\Phi^{(0)}}
  &=
  \sum_{\alpha \alpha'}
  \braket{\alpha}{\phi_{i+1}}
  \braket{\phi_i}{\alpha'}
  \lt(
    a_\alpha^\dag a_{\alpha'} \ket{\Phi^{(0)}}
  \rt)
\end{align}
Then $ a_\alpha^\dag a_{\alpha'} \ket{\Phi^{(0)}} $ represents a determinant where the diagonal of the $\alpha'$ column is zeroed and row $\alpha$ is 1.
Columns $\alpha' > N_\mtx{occ}$ are not affected, which is good because they don't exist.

The values of the off-diagonal are easily $
  \braket{\alpha}{\phi_{i+1}}
  \braket{\phi_i}{\alpha'}
$, because these are the only contributions to it.
So:
\begin{align}
  \label{eq:offdiag}
  \Phi^{(1)}_{\alpha \alpha'}
  &=
  \braket{\alpha}{\phi_{i+1}}
  \braket{\phi_i}{\alpha'}
  \qquad
  :\alpha \ne \alpha'
\end{align}
The values on the diagonal are harder to compute directly, because they get some weight for every $
    a_\alpha^\dag a_{\alpha'} \ket{\Phi^{(0)}}
$ that don't touch that column.
Probably the easiest way to get the diagonal is to enforce normalization of the columns, which represents the normalization of the orbitals that are occupied:
\begin{align}
  \label{eq:diag}
  \Phi_{\alpha \alpha}
  &=
  1 
  -
  \sum_{\alpha' \ne \alpha}
  \Phi_{\alpha' \alpha}^2
\end{align}
Eqns. (\ref{eq:offdiag}) and (\ref{eq:diag}) are sufficient to compute the singles excitation in the LO basis.

\subsection{Checking that the excitation can be done safely inside an active region}

The above works out well only if the active space LOs can represent the ground and excited state. 
One simple check for this is to check the normalization in the active space for both orbitals.
\begin{align}
  &
  \sum_{\alpha \in \mca{A}}
  \braket{\alpha}{\phi_i}^2
  = 1
  &
  &
  \sum_{\alpha \in \mca A}
  \braket{\alpha}{\phi_{i+1}}^2
  = 1,
  &
\end{align}
where $\alpha \in \mca{A}$ means LOs only in the active space.
If this doesn't hold, then the excitation requires orbitals that are frozen, and hence cannot be expressed correctly within the active space you've chosen.

\end{document}

